\section{Some ideas for a model}

It's a bit freestyle here, and we shall reconnect with theories in cognitive science, but the idea, is the following. 

There are two types of behaviors leading to their own dynamics
\begin{enumerate}
\item Optimization by small iterative steps
\item Disruptive innovations
\end{enumerate}
  
The former works almost as Bayesian learning, by the progressive integration of knowledge step by step (with some small try-and-fail), while the latter consists in jumping to new new places in the non-linear optimization space. However, the jump is not random, but builds on former experience either trying to {\it integrate} many former solutions found beforehand (yellowish horizontal stripes), or on the contrary, trying to find new solutions orthogonal to as many former solutions as possible (blueish horizontal strips). In both cases, we see that the outcome -- in terms of distance to the true model -- can be positive or negative.

\begin{table} 
    \begin{tabular}{ c c c }
         & positive outcome(improvement) & negative outcome(model worse than the best solution achieved so far) \\ 
        integrative model & X  & X \\ 
        orthogonal model & X &  X \\ 
    \end{tabular}
\label{tab:table}
\end{table}

From Figure \ref{fig:matrices}, it looks like for the game played in this experiment, radical changes have highest impact, and small changes seem to have negligible effects (it might be a result of the averaging formula employed). We shall therefore focus only on disruptive changes, their types (as described in Table \ref{tab:table}), their timing of these actions, and the resulting outcome.

Here are a few bullet points:
\begin{itemize}
\item Compute conditional probabilities of positive (resp.negative) outcome conditioned on {\it integrative} or {\it orthogonal} innovation. 
\item Look at timing of actions (not clear yet how to do this).
\end{itemize}







  
  
  