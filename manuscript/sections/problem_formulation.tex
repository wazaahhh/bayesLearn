\section{Problem Formulation}


\subsection{Distributions of JS-distance and Waiting Times}


jump size $\Delta r$ and waiting time $\Delta t$

\begin{equation}
P(R > \Delta r) \sim |\Delta r|^{-\alpha}, ~~with~~\alpha \approx 0.1,
\end{equation}

and

\begin{equation}
P(R > \Delta t) \sim |\Delta t|^{-\beta}, ~~ with~~  1< \beta < 2
\end{equation}

\subsection{Continuous Time Random Walk (CTRW)}

continuous-time random walk (CTRW)  $\rightarrow$ is a generalization of a random walk where the wandering particle waits for a random time between jumps. It is a stochastic jump process with arbitrary distributions of jump lengths and waiting times.[1][2][3] More generally it can be seen to be a special case of a Markov renewal process.

\begin{equation}
\psi(\Delta r,\Delta t)=P(\Delta r)P(\Delta t)
\end{equation}

with $P(\Delta r)$ and $P(\Delta t)$ are assumed un-correlated.





\subsection{Jensen-Shannon Distance}

JS distance : square root of the Jensen-Shannon (JS) divergence

Illustration: Figure \ref{fig:decay}

\begin{equation}
\label{JS-divergence}
JS-Divergence
\end{equation}

\begin{equation}
\label{JS-distance}
JS-Distance
\end{equation}

\subsection{Power Law Decay}

Decay  of Jensen-Shannon Distance (JSD) 

\begin{equation}
\label{power_law_decay}
JSD(t) = C \cdot t^{-\alpha},
\end{equation}

with $\alpha = 0.09$ and $C$ a constant, specific to the $simple$ and $complex$ models

 
\subsection{Stepwise Jumps}

$\rightarrow$ distribution of jumps sizes

Figure \ref{fig:jump_sizes}



\subsection{Memory Effects / Waiting Times}

Figure \ref{fig:waiting_times}


\subsection{Reuse of former configurations}



\subsection{Formulation of reuse, with memory}

