Recent research \cite{baronchelli2013levy} has suggested that cognitive mental search patterns \cite{rhodes2007human,radicchi2012rationality,radicchi2012evolution} may be inherited from typical foraging and mobility patterns of animals \cite{viswanathan1996levy,ramos2004levy,reynolds2007displaced} and humans \cite{gonzalez2008understanding,song2010mode\
lling,rhee2011levy}. In particular, patterns suggest that cognitive processes in abstract spaces are similar to food search algorithms of hunter-gatherers, modeled as L\'evy flights \cite{brown2007levy,raichlen2014evidence}. Here, we study the mental search trajectories of
individuals who have been asked to reverse engineer the conditional probabilities of 3-variable and 4-variable stochastic processes, as Bayesian
networks (48 participants per treatment) \cite{steyvers2003inferring,pearl2009causality}. We find an important contrast with the random trajectories predicted by L\'evy flight and random walk models, optimized for search of sparse targets %\cite{viswanathan1999optimizing,edwards2007revisiting,song2010modelling\
,viswanathan2011physics}.  
Human mental search exhibits temporal and spatial regularity, characterized by a time dependent distance between two consecutive solutions proposed by each individual.  Additionally, we find an unexpected tendency to return to previously tested solutions.  Both, contribute to a sub-optimal exploration of the
problem space, which in turn slows the speed of approaching the target solution.  Our results
suggest that inherent cognitive limitations hinder efficient exploration of complex abstract spaces.  These limitations appear to be hard-wired in our brain, and may stem from previously evolutionary fit hunting strategies that are inefficient for solving hard problems in modern environments. 
