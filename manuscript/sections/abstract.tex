While in cognitive sciences much research has been done about how humans learn in relatively simple problems, much less is known on the learning process associated with more complex environments. Here, we present the results of an experiment in which participants learn causal structures of varying complexity. We find that the rate of learning follows a power law decay in both the simple and complex cases where accuracy remains by a constant amount lower for the complex than for the simple case. The power law decay is very slow compared with the learning curve of fully rational computational agents. To optimize with recall and attention constraints, human participants resort to more or less explorative belief updates. The interplay of explorations into new solution subspaces and their integration with past experience for learning involves a subtle trial-and-error process, driven by timely explorative belief updates. The outcomes of these ``wild" beliefs are then either rejected or partially integrated with past experience. The slow power law decay of human learning stems on one hand from the heavy-tailed learning process, made of few wild belief updates combined with some integrative exploitation of past beliefs, and on the other hand from the heavy tailed distribution of waiting times between belief updates.  The latter process is reminiscent of human priority queueing processes and their mappping into a problem of economy of time as a non-storable scarce resource. We find that wild belief updates occur more frequently and waiting times are more heavy-tailed in the 4-node complex cases than in the 3-node simple cases. The increased difficulty forces more exploration and increases mental task loading to integrate outcomes of belief updates. Our findings contribute to a more general understanding of how humans learn about complex systems, and suggest that more complex problems require more exploration and additional learning time.