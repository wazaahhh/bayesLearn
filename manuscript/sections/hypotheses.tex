\section{Research Hypothesi(e)s}
People tackling hard problems face a tension between testing and updating their beliefs from parameters and structures stored in their memory (i.e., {\it exploitation} or {\it recombination of mental structures}), and taking action to {\it explore} and update their beliefs from not previously available mental structure. Taking such action is cognitively equivalent to the exploration of unknown territories by pioneers in the physical world, or more directly, to scientists asking and then testing new hypotheses. The former {\it exploitation} approach may bring improvement toward the solution but it is limited to a {\bf convex combination} of previously tried solutions. The latter approach carries higher potential risks (behind the hill a leopard might be lurking, or research funds might be wasted on finding nothing of interest) as well as higher potential returns (there might be an unmeasurable treasure hidden behind the hill, or a cure for cancer might be found), but whether exploration brings improvement towards the solution at some given moment or not, this strategy expands the cognitive frontier. Below we refer to the convex hull of previously explored solutions as the {\it cognitive frontier}.  Once a new portion of the solution space has been explored, the attempted proposed model is then stored into memory and may be recombined, later on, with other proposed models, in proportion with its believed usefulness. \\

